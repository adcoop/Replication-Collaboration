\documentclass[12pt,letterpaper]{article}
\usepackage[utf8]{inputenc}
\usepackage{amsmath}
\usepackage{amsfonts}
\usepackage{amssymb}
\usepackage{graphicx}
\usepackage[left=2cm,right=2cm,top=2cm,bottom=2cm]{geometry}
\author{Zenobia Chan, Alicia Cooperman, and Lauren Young}
\title{Green and Vasudevan (2015) Pre-replication Plan}
\begin{document}
\maketitle
This is a really interesting field experiment with strong treatment effects suggesting that anti-vote-buying radio campaigns do decrease vote share for vote-buying parties. In our replication, we fill focus on different ways of understanding the substantive effect, particularly heterogeneous treatment effects, the role of multiple radio stations covering certain regions, and alternative codings of vote-buying parties.\\

Our initial replication approach will be as follows:
\begin{itemize}
\item Heterogeneous effects by underlying characteristics (rural/urban, SC/ST caste, including external data on corruption probes)
\item Multiple/continous treatment and notion of partial vs. full coverage
\item Alternative coding of corrupt parties (quantifying agreement between journalists in the same location; distance measure of unanimity)
\item Understanding the effects: are there certain demographics driving these findings? Certain state? What is vote buying correlated with in general? Are some voters switching to smaller third-place protest parties, while main vote-buying parties still receive majority votes? What does this tell us about vote-buying and politics in India?
\end{itemize}

\end{document}